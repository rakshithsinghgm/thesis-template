\chapter{Abstract}

We are currently experiencing the onset of the fourth industrial revolution, which is bringing about significant changes in our world through advancements in computing capacity, connectivity, big data, and artificial intelligence\cite{velarde2020artificial}. One area that has been particularly impacted by these changes is autonomy and machine learning. The development of technologies such as self-driving cars\cite{bertoncello2015ten} and large multi-modal models\cite{zhao2023survey} is opening up new markets and engaging a vast number of consumers. In the maritime industry, there is a growing interest in leveraging these advancements to create autonomous solutions for vessels\cite{gu2021autonomous} that can aid with specific operations such as auto-docking, automated short-sea-shipping, captain-assistance systems for obstacle avoidance, blind spot monitoring, and swimmer detection. This emerging field is known as Marine Autonomy. \\

Any autonomous solution requires at least two or three of these building blocks (based on the "level" of autonomy as defined by SAE \cite{on2014taxonomy}) - a Perception System to "sense" the environment, a decision-making system, or path planner to "think" about the next possible action to take, and a Control System to "act" on the said action\cite{bagloee2016autonomous}. Machine learning plays an important role in perception systems. Computer vision models based on deep learning machine learning algorithms are used for various tasks such as object detection, scene interpretation, and scene segmentation\cite{voulodimos2018deep}. These models play a critical role in any Perception System stack by helping autonomous vehicles scan and develop a better understanding of the surrounding environment.\cite{hane20173d} \\

Developing good machine learning models requires large amounts of high-quality "training data" (for vision based applications, training data is in the form of images and text annotations).\cite{radiuk2017impact} For context, Meta AI's Segment Anything Model (SAM), was trained on a dataset of 11 million images and over 1 billion annotations.\cite{kirillov2023segment} These images should be taken from multiple angles, object perspectives, lighting scenarios, etc, to develop a rich and diverse dataset. The complexity is even greater for deep learning algorithms, which require enormous amounts of training data. Developing such large datasets often involves collaboration between industry and academia, or even the government, as well as crowd-sourcing annotations. If images are commercially labeled on a large scale through a labeling platform, the overall cost of the labeling project can be reduced due to the economics of scale. \\

Collecting data in novel or constrained environments like marine environments can be challenging and expensive. Such environments are filled with various types of boats, stationary objects, and structures, and the cost of data collection quickly adds up due to logistics such as acquiring data collection vessels and vessel captains, and configuring sensors. Therefore, decisions on the category of objects to target for data collection become critical, since changes in these categories or ontology can result in significant rework and capital loss. \\ 

The aim of this thesis is two-fold. Firstly, we propose an ontology for autonomous marine vessels that encompasses various types of commonly found vessels and marine obstacles. We will describe the thought process and considerations involved in developing such an ontology. Our ontology work was primarily inspired by COLREGS, which refers to the International regulations for preventing collisions at Sea developed by the International Maritime Organization (IMO). Secondly, we will describe and demonstrate the workflow to develop an object detection model trained on a custom maritime image dataset that follows the above ontology and test it through integration in the perception system of an autonomous marine vessel. Our work hopes to develop a base Marine Perception ontology that can enable further research in autonomous marine vessels.